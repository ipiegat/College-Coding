\documentclass[12pt]{article}
\usepackage{geometry}
\usepackage{amsmath}
\geometry{a4paper, margin=1in}

\title{Cryptography Cheat Sheet}
\author{Isaac Piegat}
\date{\today}

\begin{document}

\maketitle

\section*{Cryptographic Hash Functions (CRHFs)}
\subsection*{Description}
A hash function maps an input of arbitrary size to a fixed-size output, ensuring unique "fingerprints" for data. A \textbf{CRHF} specifically ensures collision resistance.

\subsection*{Conditions}
\begin{itemize}
    \item \textbf{Collision Resistance:} Hard to find two inputs \(x \neq x'\) such that \(h(x) = h(x')\).
    \item \textbf{Second-Preimage Resistance:} Given \(x\), hard to find \(x' \neq x\) where \(h(x) = h(x')\).
    \item \textbf{Preimage Resistance:} Given \(y\), hard to find any \(x\) where \(h(x) = y\).
\end{itemize}

\subsection*{Weaknesses/Common Attacks}
\begin{itemize}
    \item \textbf{Collision Attacks:} Exploiting weaknesses in the hash function to find two inputs with the same output (e.g., MD5, SHA-1).
    \item \textbf{Length Extension Attacks:} Extending valid hashes without knowing the original input.
    \item \textbf{Birthday Attacks:} Using the birthday paradox to find collisions faster than brute force.
\end{itemize}

\section*{One-Way Functions (OWFs)}
\subsection*{Description}
A function \(f(x)\) that is easy to compute but computationally infeasible to invert. Used for securing data such as passwords.

\subsection*{Conditions}
\begin{itemize}
    \item \textbf{Easy to Compute:} \(f(x)\) is efficient to evaluate.
    \item \textbf{Hard to Invert:} Given \(y = f(x)\), it’s computationally infeasible to find \(x\).
\end{itemize}

\subsection*{Weaknesses/Common Attacks}
\begin{itemize}
    \item \textbf{Brute-Force Attacks:} If the input space is small, attackers can try all possibilities.
    \item \textbf{Weak Randomness:} Predictable inputs compromise security.
    \item \textbf{Structural Weaknesses:} Poorly designed OWFs can have shortcuts for inversion.
\end{itemize}

\section*{Hashes}
\subsection*{Description}
Hash functions compress data into fixed-size values for integrity, signatures, and authentication. Cryptographic hashes are a secure subset used in protocols.

\subsection*{Conditions}
\begin{itemize}
    \item \textbf{Fixed Output Size:} Maps any input to a specific-length output.
    \item \textbf{Deterministic:} Same input always produces the same output.
    \item \textbf{Avalanche Effect:} A small change in input drastically changes the output.
\end{itemize}

\subsection*{Weaknesses/Common Attacks}
\begin{itemize}
    \item \textbf{Collision Attacks:} Finding two inputs with the same hash.
    \item \textbf{Rainbow Table Attacks:} Precomputed hash chains to reverse hashes.
    \item \textbf{Timing Attacks:} Using timing information to guess the hash process or input.
\end{itemize}

\section*{Shared Key Protocols (SKPs)}
\subsection*{Description}
Protocols used to securely establish a shared secret key between two parties over an insecure channel. The key is then used for encryption or authentication.

\subsection*{Conditions}
\begin{itemize}
    \item \textbf{Authentication:} Each party verifies the identity of the other.
    \item \textbf{Confidentiality:} The key remains secret during exchange.
    \item \textbf{Integrity:} Messages exchanged during the protocol must not be altered.
\end{itemize}

\subsection*{Weaknesses/Common Attacks}
\begin{itemize}
    \item \textbf{Replay Attacks:} Reusing intercepted messages to impersonate one party.
    \item \textbf{Man-in-the-Middle (MitM) Attacks:} Intercepting and modifying communication.
    \item \textbf{Weak Nonces or Counters:} Predictable or reused values compromise security.
    \item \textbf{Reflection Attacks:} Tricking parties into processing their own messages.
\end{itemize}

\section*{Message Authentication Codes (MACs)}
\subsection*{Description}
A Message Authentication Code (MAC) is a cryptographic tool used to ensure the authenticity and integrity of a message. It uses a shared secret key \( K \) known only to the communicating parties.

\subsection*{How It Works}
Given a message \( m \) and a shared key \( K \):
\[
\text{MAC} = \text{MAC}_K(m)
\]
The recipient can verify that:
\begin{itemize}
    \item \( m \) has not been tampered with (integrity).
    \item \( m \) originates from someone who knows \( K \) (authenticity).
\end{itemize}

\subsection*{Conditions for a Secure MAC}
\begin{itemize}
    \item \textbf{Key-Dependent:} The MAC must depend on the secret key \( K \). Without \( K \), an attacker should not be able to forge a valid MAC.
    \item \textbf{Unforgeability:} It must be computationally infeasible to generate a valid MAC without knowing \( K \).
    \item \textbf{Resistance to Replay Attacks:} Often achieved by including a nonce or timestamp with the message.
\end{itemize}

\subsection*{Types of MACs}
\begin{itemize}
    \item \textbf{HMAC:} Hash-based MACs use a cryptographic hash function (e.g., SHA-256) combined with a secret key.
    \item \textbf{CBC-MAC:} Cipher Block Chaining MACs are based on block ciphers and operate on fixed-size message blocks.
    \item \textbf{GMAC:} A MAC based on Galois Field arithmetic, used in authenticated encryption schemes.
\end{itemize}

\subsection*{Applications}
\begin{itemize}
    \item Authenticating messages in communication protocols (e.g., SSL/TLS, IPsec).
    \item Ensuring data integrity in storage systems.
    \item Preventing tampering in distributed systems.
\end{itemize}

\subsection*{Weaknesses/Common Attacks}
\begin{itemize}
    \item \textbf{Key Compromise:} If \( K \) is leaked, the attacker can forge valid MACs.
    \item \textbf{Length Extension Attacks:} A poorly implemented MAC based on hash functions can be vulnerable if intermediate states of the hash are exposed.
    \item \textbf{Reused Nonces:} If a MAC includes a nonce that is reused, it can allow replay attacks.
\end{itemize}

\section*{Combining Encryption and Authentication}
There are three primary methods to combine encryption and authentication in cryptographic protocols:

\subsection*{Encrypt-then-MAC (EtM)}
\begin{itemize}
    \item \textbf{How It Works:} First, the plaintext is encrypted to produce the ciphertext. Then, a MAC is calculated over the ciphertext using a secret key.
    \item \textbf{Order of Operations:}
    \[
    \text{Ciphertext} = E_k(\text{plaintext})
    \]
    \[
    \text{MAC} = \mathrm{MAC}_k(\text{Ciphertext})
    \]
    \item \textbf{Advantages:} 
    \begin{itemize}
        \item Provides strong security.
        \item Detects tampering with the ciphertext before decryption.
    \end{itemize}
    \item \textbf{Example Use Case:} IPsec.
\end{itemize}

\subsection*{MAC-then-Encrypt (MtE)}
\begin{itemize}
    \item \textbf{How It Works:} First, a MAC is computed on the plaintext. Then, the plaintext and its MAC are encrypted together.
    \item \textbf{Order of Operations:}
    \[
    \text{MAC} = \mathrm{MAC}_k(\text{plaintext})
    \]
    \[
    \text{Ciphertext} = E_k(\text{plaintext} \| \text{MAC})
    \]
    \item \textbf{Advantages:}
    \begin{itemize}
        \item Protects both the plaintext and the MAC.
    \end{itemize}
    \item \textbf{Disadvantages:}
    \begin{itemize}
        \item Vulnerable if decryption is performed before verifying the MAC.
    \end{itemize}
    \item \textbf{Example Use Case:} SSL/TLS (pre-TLS 1.3).
\end{itemize}

\subsection*{Encrypt-and-MAC (E\&M)}
\begin{itemize}
    \item \textbf{How It Works:} The plaintext is both encrypted and authenticated independently.
    \item \textbf{Order of Operations:}
    \[
    \text{Ciphertext} = E_k(\text{plaintext})
    \]
    \[
    \text{MAC} = \mathrm{MAC}_k(\text{plaintext})
    \]
    \item \textbf{Advantages:}
    \begin{itemize}
        \item Simpler conceptually; failures in one process (encryption or MAC) do not compromise the other.
    \end{itemize}
    \item \textbf{Disadvantages:}
    \begin{itemize}
        \item Less efficient because encryption and authentication are independent.
    \end{itemize}
    \item \textbf{Example Use Case:} Rarely used in modern protocols.
\end{itemize}

\section*{MACs vs. Hash Functions}
\subsection*{Message Authentication Codes (MACs)}
\begin{itemize}
    \item \textbf{Description:} A MAC binds a message to a shared secret key \( K \), providing integrity and authenticity.
    \item \textbf{Purpose:} Ensures that the message is from an authenticated sender and has not been tampered with.
\end{itemize}

\subsection*{Hash Functions}
\begin{itemize}
    \item \textbf{Description:} A hash function produces a fixed-length digest of a message but does not use a secret key.
    \item \textbf{Limitation:} Hash functions do not provide authenticity; anyone can compute a valid hash for a modified message.
\end{itemize}

\subsection*{Why MACs Are Superior}
\begin{itemize}
    \item \textbf{Key Binding:} A MAC ensures that the message is bound to the shared secret \( K \), preventing forgery.
    \item \textbf{Tamper Resistance:} An attacker cannot recompute a valid MAC without knowing \( K \).
    \item \textbf{Man-in-the-Middle Defense:} MACs resist MitM attacks, where a hash function might fail due to lack of key-dependency.
\end{itemize}

\section*{Modular Arithmetic}
\subsection*{Basic Operations}
Modular arithmetic operates within the range \( \{0, 1, \dots, n-1\} \) for a modulus \( n \). The key operations include:

\subsubsection*{Addition Modulo \( n \)}
\[
(a + b) \mod n = ((a \mod n) + (b \mod n)) \mod n
\]

\subsubsection*{Multiplication Modulo \( n \)}
\[
(a \cdot b) \mod n = ((a \mod n) \cdot (b \mod n)) \mod n
\]

\subsubsection*{Subtraction Modulo \( n \)}
\[
(a - b) \mod n = ((a \mod n) - (b \mod n)) \mod n
\]

\subsection*{Distributive Property}
For any integers \( a, b, c \):
\[
(a \cdot (b + c)) \mod n = ((a \cdot b) + (a \cdot c)) \mod n
\]

\section*{Fermat's Little Theorem}
\textbf{Statement:} If \( p \) is a prime number and \( a \) is an integer such that \( p \not\mid a \), then:
\[
a^{p-1} \equiv 1 \mod p
\]

\section*{Euler's Totient Function}
\subsection*{Definition}
The totient function \( \phi(n) \) counts the integers between \( 1 \) and \( n \) that are coprime to \( n \).

\subsection*{Formula for \( \phi(n) \)}
For \( n = p_1^{e_1} \cdot p_2^{e_2} \cdot \dots \cdot p_k^{e_k} \):
\[
\phi(n) = n \cdot \prod_{i=1}^{k} \left(1 - \frac{1}{p_i}\right)
\]

\section*{RSA Key Generation}
\subsection*{Steps to Generate Keys}
1. Select two large prime numbers \( p \) and \( q \).
2. Compute \( n = p \cdot q \).
3. Compute \( \phi(n) = (p-1) \cdot (q-1) \).
4. Choose \( e \) such that \( 1 < e < \phi(n) \) and \( \gcd(e, \phi(n)) = 1 \).
5. Compute \( d \) such that:
\[
e \cdot d \equiv 1 \mod \phi(n)
\]
6. Public key: \( (e, n) \). Private key: \( d \).

\subsection*{Encryption and Decryption}
\begin{itemize}
    \item \textbf{Encryption:} Given message \( m \):
    \[
    c \equiv m^e \mod n
    \]
    \item \textbf{Decryption:} Recover \( m \) from \( c \):
    \[
    m \equiv c^d \mod n
    \]
\end{itemize}


\end{document}
