\documentclass[12pt]{article}
\usepackage{amsmath,amssymb,epsfig,boxedminipage,helvet,theorem,endnotes,version}
\usepackage{mathtools}
\newcommand{\remove}[1]{}
\setlength{\oddsidemargin}{-.2in}
\setlength{\evensidemargin}{-.2in}
\setlength{\textwidth}{6.5in}
\setlength{\topmargin}{-0.8in}
\setlength{\textheight}{9.5in}


\newtheorem{theorem}{Theorem} 
\newtheorem{definition}{Definition}

\newcommand{\gen}{\mbox{\tt Gen}}
\newcommand{\enc}{\mbox{\tt Enc}}
\newcommand{\dec}{\mbox{\tt Dec}}

\newcommand{\sol}{{\bf Solution}: }

\newcommand{\zo}{\{0,1\}}
\newcommand{\zoell}{\{0,1\}^\ell}
\newcommand{\zon}{\{0,1\}^n}
\newcommand{\zok}{\{0,1\}^k}

%% eg \priv{eav}{n,b} for def 3.9 style, or \priv{eav}{n} for def 3.8
\newcommand{\priv}[2]{${\tt Priv}_{A,\Pi}^{\tt #1}(#2)$}

\newcommand*\concat{\mathbin{\|}}

\newcommand\xor{\oplus}

\newcommand{\handout}[2]{
\renewcommand{\thepage}{\footnotesize CSE 3400/CSE 5850, #1, p. \arabic{page}}
\begin{center}

\noindent
{\bf UConn, School of Computing}

\noindent
{\bf Fall 2024}

\noindent
{\bf CSE 3400/CSE 5850: Introduction to Computer and Network Security \\ / Introduction to Cybersecurity}
\end{center}

\begin{center}
{\Large #1}
\end{center}
}



\begin{document}

\handout{Assignment 3}{}

\noindent
{Instructor: Prof.~Ghada Almashaqbeh}
\newline
{Student: Isaac Piegat}
\newline
\noindent
{Posted: 9/28/2024}
\newline
{Submitted: 10/7/2024}
\noindent
\newline
{Submission deadline: 10/5/2024, 11:59 pm} \\\\

\noindent{\bf Notes:} 
\begin{itemize}
\item Solutions {\bf must be typed} (using latex or any other text editor) and must be submitted as a pdf (not word or source latex files).

\item This homework has a \textbf{shorter late days allowance} than usual. It will be \textbf{only 2 days} to allow us to post the key solution before midterm test 1. Thus, if you still have free late days (2 or more), you can use up to 2 days, and if not, there will be a deduction for the late day. After 2 days from the deadline no late submissions will be accepted (and the key solution will be posted).

\item In all of the below, if the scheme is insecure then provide an attack against it and analyze its success probability/attacker's advantage. If the scheme is secure, just provide a convincing argument along with the attacker's advantage (formal security proofs are not required).
\\
\end{itemize}


\noindent{\bf Problem 1 [60 points]\\}
This problem is about encryption modes.
\begin{enumerate}
\item Rafa claims that if we modify the ECB mode as follows, it will become CPA secure: As before, a message is divided into blocks, so a message $m$ would be $m = m_0 \concat m_1 \concat \dots \concat m_w$ (where $w$ is an integer). For each message, we generate a fresh random string $r$ of length $n$ bits, and we encrypt the message as $E_k(m) = (E_k(m_1 \concat r), E_k(m_2 \concat r), \dots, E_k(m_w \concat r))$. Is Rafa's claim true? why? 
\\\\Rafa is correct. Concatonating each message block, each with their own unique random string of length n, means the
encryption mode will become non-deterministic because even if a block like $m_1$ is reused, the random string concatonated onto
the end would generate different outputs in the encryption function. 
\item Coco wants to output one of the intermediary pads (say $pad_1$) in the OFB mode as part of the ciphertext of the messages she sends to Rafa. Does this modification preserve the CPA security of the OFB mode? why? 
\\\\This modification does not preserve the security of the OFB mode because once Rafa knows the pad, Rafa can then
derive the plaintext by xoring the ciphertext and now known pad. This information compromises the CPA security of the OFB mode. 
\item In the CTR mode, Alice and Bob decided to sync their counters via two step increments instead of one as in the original CTR mode we studied in class. That is, for a message $m = m_1 \concat m_2 \concat \dots \concat m_w$ (where $w$ is an integer), for $m_1$ the counter value would be $s$ (the random initial value of the counter), for $m_2$ the counter value would be $s+2$, for $m_3$ the counter value would be $s+4$ and so on. Does this modification impact the CPA security and correctness of the CTR encryption mode? why?
\\\\As long as the value of s is unique and not reused, the security and correctness is not impacted. Two step increments does
nothing to change the uniqueness of s, and seeing how it is not reused because s is unique, the modification does not impact
security or correctness. 
\item What is the effect of the following ciphertext reordering/dropping/corruption on correctness of decryption for each of the OFB, CBC and CTR modes (note that for CTR mode $c_0 = s$ is the initial value of the counter, while it is IV for CBC and CTR modes). In all cases you have to justify your answers: 
\begin{enumerate}
    \item Alice sent Bob ciphertext $c_0, c_1, c_2, c_3, c_4, \dots, c_w$, which was received by Bob as $c_0, c_1, c_3, c_4, \dots, c_w$ (so $c_2$ was dropped). 
\\\\OFB - Only $m_2$ is effected as it is lost. All other plaintexts are accurate because the keystream is only generated the IV.
\\\\CTR - Only $m_2$ is effected as it is lost. All other ciphertexts are accurate because the counters for $c_3, c_4, ...$ are independent.
\\\\CBC - Only $m_1$ is not effected. Each block of ciphertext is dependent on the previous block, thus all blocks after $c_2$ are corrupted. 
\newline    
    \item Alice sent Bob ciphertext $c_0, c_1, c_2, c_3, c_4, \dots, c_w$, which was received by Bob as $c_0, c_2, c_4, c_3, c_1, \dots, c_w$ (so there is a reordering of the ciphertext over the channel that Bob does not know about).
\\\\OFB and CTR - Unaffected. All plaintexts will be accurate but in the wrong order because the decryption process does not rely on the previous cipher text block.
\\\\CBC - All plaintexts except $m_1$ will be corrupted. CBC requires the previous block in order to decrypt the ciphertext correctly. Messing up the order would then mess up the decryption, corrupting all plaintexts except $m_0$ in this case.
\newline
    \item Alice sent Bob ciphertext $c_0, c_1, c_2, c_3, c_4, \dots, c_w$. Bob received the ciphertext (all of it in the same order) but with the last two bits of $c_0$ flipped.
\\\\OFB and CTR - Only $m_0$ is corrupted. All ciphertexts do not rely on each other for decryption, thus the rest are correct.
\\\\CBC - The incorrect bits in $c_0$ would corrupt $m_0$ and $m_1$ as the decryption process relies on the ciphertext of the previous block. All other plaintexts are correct as $c_1, c_2, ...$ are correct. 
    \item Alice sent Bob ciphertext $c_0, c_1, c_2, c_3, c_4, \dots, c_w$. Bob received the ciphertext (all of it in the same order) but with the left half of $c_3$ is flipped.
\\\\OFB and CTR - Only $m_3$ is corrupted. All ciphertexts do not rely on each other for decryption, thus the rest are correct.
\\\\CBC - The incorrect bits in $c_3$ would corrupt $m_3$ and $m_4$ as the decryption process relies on the ciphertext of the previous block. All other plaintexts are correct as $c_1, c_2, c_4, ...$ are correct. 
\end{enumerate} 


\end{enumerate}
\noindent\textbf{Note:} This problem has a bonus of 10 points.\\
% ***********************************************************************************************

\vspace{12pt}
\noindent{\bf Problem 2 [60 points]\\}
Let $G:\zo^{n/2} \to \zo^{n}$ be a secure PRG, and $F:\zo^n \times \zo^n \to \zo^n$ be a secure PRF. For each of the following MAC constructions, state whether it is a secure MAC and justify your answers.
\begin{enumerate}
\item Given message $m \in \zo^{n/2}$, compute the tag as $MAC_k(m) = F_k(G(m)) \concat LS2B(m)$, where $LS2B(m)$ is the last 2 bits of $m$.
\\\\The MAC is secure because even though the attacker can identify the last two bits of the tag, this gives the attacker a non-neligible advantage in forging the tag of \(1/2^{n-2}\). This number is still ~ 0.
\item Given message $m \in \zo^{n}$, compute $y = F_k(m)$, parse $y = y_0 \concat y_1$ such that $|y_0| = |y_1| = n/2$, then compute the tag as $G(y_0)\oplus G(y_1)$.
\\\\This MAC is secure. The output would be of length $2n$ as you would pass both halves through the secure PRG (G) effectively doubling and adding two halves ($1/2*2 + 1/2*2 = 2$). The randomness of the PRG would then give the attacker no advantage, thus $(1/2^n) ~ 0$. 
\item Given message $m \in \zo^{3n}$, parse $m$ as $m = m_0 \concat m_1 \concat m_2$ such that $|m_0| = |m_1| = |m_2| = n$. Compute the tag as $MAC_k(m) = F_k(m_0) \concat F_k(m_1 \oplus m_2)$.
\\\\This MAC construction is not secure because the attacker can reorder the message, effectively changing the message. If the attacker swapped the positions of $m1$ and $m2$, the tag will still be valid but the message would be different. This means the attacker can create a correct forgery 100  percent of the time, thus giving the attacker a non-negligible advantange of 1.  
\item A variation of the CMAC construction: Assume the message $m$ to be of an even number of blocks (so it is a VIL whose length can be any even number of blocks). We compute a tag as $CMAC_k(m) = CBC-MAC_k(\frac{L(m)}{2} \concat m_1 \concat \dots \concat m_{L(m)/2}) \concat CBC-MAC_k(\frac{L(m)}{2} \concat m_{(L(m)/2) + 1} \concat \dots \concat m_{L(m)})$, where $\frac{L(m)}{2}$ is a block representing half the length of the message $m$.
\\\\The CMAC is not secure because if the attacker quiered the first half of the message they wish to send, and repeated that for the second half, that would give the attacker half the key which contains a repeat of the key for that half of the message. He can do this again with the other half of the message and would get the key for that half, allowing them to forge the key for every message.
\end{enumerate} 

\noindent\textbf{Note:} This problem has a bonus of 10 points.
\end{document} 