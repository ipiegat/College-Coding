\documentclass{article} % Defines the type of document (e.g., article, report, book)
\usepackage[utf8]{inputenc} % Handles input encoding

\title{CSE3504 Homework 1} % Title of your document
\author{Isaac Piegat} % Author's name
\date{\today} % Date (you can also set it manually, like \date{September 10, 2024})

\begin{document} % Start of the document
\maketitle % Generates the title section



\section{Problem 1:}
\subsection{Let S = {1, 2, ,,,, 100}. Define \(E_2\) as the event that a number is divisible by 2, and \(E_3\) as event that the number is divisible by 3.}
\begin{itemize}
    \item The cardinality of event \(E_2\) is \(100/2 = 50\) and event \(E_3\) is \(100/3 = 33\).
    \item Even numbers divisible by 3 are also divisible by 2, thus half of all numbers divisible by 3 
    are also divisible by 2. This means the cardinality between the intersection of \(E_2\) and \(E_3\) 
    is \(100/3 * 1/2 = 100/6 = 16\).
\end{itemize}

\section{Problem 2:}
\subsection{Two teams A and B play a soccer match, and we are interested in the winner. The sample
space can be defined as: S = {a, b, d} where “a” shows the outcome that A wins, “b” shows
the outcome that B wins, and “d” shows the outcome that they draw. Suppose that we know
that the probability that A wins is P(a) = 0.5 and the probability of a draw is P(d) = 0.25.}
\begin{itemize}
    \item The probability that B wins is \(P(b) = 1 - P(a) - P(d) = 1 - .5 - .25 = .25\).
    \item The probability that B wins or a draw occurs is \(P(b or d) = 1- P(a) = .5\).
\end{itemize}

\section{Problem 3:}
\subsection{Three factories make .20, .30, and .50 of the computer chips for a company. The
probability of a defective chip is 0.04, 0.03, and 0.02 for the three factories.}

\begin{itemize}
    \item The probability that a chip is defective is \(.2*.04 + .3*.03 + .5*.02 = .027\).
    \item If a chip is defective, the chances it came from factory one is \((.04*.2)/.027 = .296\).
\end{itemize}

\section{Problem 4:}
\subsection{A password consists of six characters. These characters are chosen from the 10 digits
and 26 letters of the alphabet. Passwords are also case sensitive.}
\begin{itemize}
    \item There are \(36^6\) different combinations of passwords (lowercase and uppercase letters).
    \item Using the non-replacement formula \(N!/(N-k)!\) you get \(36!/30!\).
    \item A hacker guessing 100 million passwords per second would take \(36^6/10^8 = 21.7s\).
    \item To choose a password with a letter and a number you would have to first select from 26 letters. There are a total of 
    \(36^5\) passwords with no constraints and \(26^5\) passwords with no digits (there are 10 total digits). Thus, the
    number of valid passwords would be \(26*(36^5-26^5) = 1,263,204,800\).
    \item \(27,868,297,600/100,000,000 = 12s\).

\end{itemize}

\section{}
\subsection{}
\begin{itemize}
    \item pass

\end{itemize}
\end{document}