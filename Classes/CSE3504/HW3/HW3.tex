\documentclass{article} % Defines the type of document (e.g., article, report, book)
\usepackage[utf8]{inputenc} % Handles input encoding
\usepackage{amsmath}

\title{CSE3504 Homework 1} % Title of your document
\author{Isaac Piegat} % Author's name
\date{\today} % Date (you can also set it manually, like \date{September 10, 2024})

\begin{document} % Start of the document
\maketitle % Generates the title section



\section{Problem 1:}
Let \( X \) be a continuous random variable with the probability density function (PDF):

\[
f_X(x) = 
\begin{cases}
6x(1 - x), & 0 \leq x \leq 1, \\
0, & \text{otherwise}.
\end{cases}
\]

\subsection*{(a) Find the CDF of \( X \), \( F_X(x) \).}

The cumulative distribution function (CDF) is obtained by integrating the PDF:

\[
F_X(x) = \int_{0}^{x} 6t(1 - t) \, dt.
\]

First, expand and integrate:

\[
F_X(x) = 6 \int_{0}^{x} (t - t^2) \, dt
= 6 \left[ \frac{t^2}{2} - \frac{t^3}{3} \right]_0^x
= 6 \left( \frac{x^2}{2} - \frac{x^3}{3} \right).
\]

Thus, the CDF is:

\[
F_X(x) = 3x^2 - 2x^3.
\]

\subsection*{(b) Find \( P(X < 1/4) \) and compute the probability.}

The probability \( P(X < 1/4) \) is given by \( F_X(1/4) \):

\[
F_X(1/4) = 3\left( \frac{1}{4} \right)^2 - 2\left( \frac{1}{4} \right)^3
= 3\left( \frac{1}{16} \right) - 2\left( \frac{1}{64} \right)
= \frac{3}{16} - \frac{2}{64}
= \frac{3}{16} - \frac{1}{32}
= \frac{6}{32} - \frac{1}{32}
= \frac{5}{32}.
\]

Thus, \( P(X < 1/4) = \frac{5}{32} \).

\subsection*{(c) Find \( P(X > 1/2) \) and compute the probability.}

The probability \( P(X > 1/2) \) is given by:

\[
P(X > 1/2) = 1 - P(X \leq 1/2) = 1 - F_X(1/2).
\]

First, compute \( F_X(1/2) \):

\[
F_X(1/2) = 3\left( \frac{1}{2} \right)^2 - 2\left( \frac{1}{2} \right)^3
= 3\left( \frac{1}{4} \right) - 2\left( \frac{1}{8} \right)
= \frac{3}{4} - \frac{1}{4} = \frac{2}{4} = \frac{1}{2}.
\]

Thus,

\[
P(X > 1/2) = 1 - \frac{1}{2} = \frac{1}{2}.
\]

\subsection*{(d) Find the median of \( X \).}

The median is the value \( m \) such that \( P(X \leq m) = 0.5 \), or equivalently \( F_X(m) = 0.5 \). From part (c), we found that \( F_X(1/2) = 0.5 \), so the median of \( X \) is:

\[
m = \frac{1}{2}.
\]

\section*{Problem 2}

A large lot of marbles have diameters which are approximately normally distributed with a mean of \(1 \, \text{cm}\). One third of the marbles have diameters greater than \(1.1 \, \text{cm}\).

\subsection*{(a) Find the standard deviation of the distribution.}

We are given that one third of the marbles have diameters greater than \(1.1 \, \text{cm}\). This implies:

\[
P(X > 1.1) = \frac{1}{3}.
\]

Using the standard normal distribution, we need to find the corresponding \(z\)-score for which \(P(Z > z) = \frac{1}{3}\). From the z-table, this corresponds to:

\[
z = -0.43.
\]

Now, we use the z-score formula to find the standard deviation \( \sigma \):

\[
z = \frac{1.1 - 1}{\sigma}.
\]

Substituting the values:

\[
-0.43 = \frac{0.1}{\sigma},
\]

\[
\sigma = \frac{0.1}{0.43} \approx 0.2326.
\]

Thus, the standard deviation is approximately:

\[
\sigma \approx 0.2326 \, \text{cm}.
\]

\subsection*{(b) Proportion whose diameters are within 0.2 cm of the mean.}

We are asked to find \(P(0.8 \leq X \leq 1.2)\). First, convert the values to z-scores:

\[
P\left( \frac{0.8 - 1}{\sigma} \leq Z \leq \frac{1.2 - 1}{\sigma} \right)
= P\left( \frac{-0.2}{0.2326} \leq Z \leq \frac{0.2}{0.2326} \right)
= P(-0.86 \leq Z \leq 0.86).
\]

Using the z-table or R function to find the cumulative probabilities:

\[
P(-0.86 \leq Z \leq 0.86) = P(Z \leq 0.86) - P(Z \leq -0.86)
= 0.8051 - 0.1949 = 0.6102.
\]

Thus, the proportion of marbles with diameters within \(0.2 \, \text{cm}\) of the mean is approximately:

\[
P(0.8 \leq X \leq 1.2) \approx 0.6102.
\]

To compute this in R:

Compute the proportion within 0.2 cm of the mean
pnorm(0.2 / 0.2326) - pnorm(-0.2 / 0.2326)

\subsection*{(c) Diameter that is exceeded by 75\% of the marbles.}

We are asked to find the diameter \( d \) such that \( P(X > d) = 0.75 \). This is equivalent to finding the z-score for which \( P(Z > z) = 0.75 \), or equivalently \( P(Z \leq z) = 0.25 \). 

Using the z-table, we find that the z-score corresponding to \( P(Z \leq z) = 0.25 \) is:

\[
z = -0.6745.
\]

Now, we use the z-score formula to relate \( z \), \( d \), the mean, and the standard deviation:

\[
z = \frac{d - 1}{\sigma}.
\]

Substituting the known values:

\[
-0.6745 = \frac{d - 1}{0.2326}.
\]

Solving for \( d \):

\[
d - 1 = -0.6745 \times 0.2326 \approx -0.157,
\]

\[
d = 1 - 0.157 = 0.843.
\]

Thus, the diameter that is exceeded by 75\% of the marbles is approximately:

\[
d \approx 0.843 \, \text{cm}.
\]

\section*{Problem 3}

Transistors produced by one machine have a lifetime which is exponentially distributed with rate \( \lambda = 0.01 \, \text{hour}^{-1} \). Transistors produced by a second machine have an exponentially distributed lifetime with rate \( \lambda = 0.005 \, \text{hour}^{-1} \). A package of 12 transistors contains 4 produced by the first machine and 8 produced by the second machine.

\subsection*{(a) Probability that the lifetime of a randomly chosen transistor exceeds 100 hours.}

The probability that the lifetime \( T \) of an exponentially distributed random variable exceeds \( t \) hours is given by:

\[
P(T > t) = e^{-\lambda t}.
\]

We need to calculate the overall probability for the randomly chosen transistor, which could come from either machine. The probability that a randomly chosen transistor comes from the first machine is \( \frac{4}{12} = \frac{1}{3} \), and from the second machine is \( \frac{8}{12} = \frac{2}{3} \).

For the first machine (\( \lambda_1 = 0.01 \)):

\[
P(T_1 > 100) = e^{-0.01 \times 100} = e^{-1} \approx 0.3679.
\]

For the second machine (\( \lambda_2 = 0.005 \)):

\[
P(T_2 > 100) = e^{-0.005 \times 100} = e^{-0.5} \approx 0.6065.
\]

Now, we calculate the weighted average:

\[
P(T > 100) = \frac{1}{3} P(T_1 > 100) + \frac{2}{3} P(T_2 > 100)
= \frac{1}{3} \times 0.3679 + \frac{2}{3} \times 0.6065
= 0.4470.
\]

Thus, the probability that the lifetime of a randomly chosen transistor exceeds 100 hours is approximately:

\[
P(T > 100) \approx 0.4470.
\]


To compute this in R:

Calculate the probabilities using the exponential distribution
\newline
$P_{T1} <- exp(-0.01 * 100)
P_{T2} <- exp(-0.005 * 100)$

Compute the weighted probability
$P_100 <- (1/3) * P_T1_100 + (2/3) * P_T2_100$

\subsection*{(b) Probability that the lifetime exceeds 200 hours, given it has already exceeded 150 hours.}

For an exponential distribution, the memoryless property holds, which means that:

\[
P(T > t + s \mid T > s) = P(T > t).
\]

In this case, we want to find \( P(T > 200 \mid T > 150) \), which simplifies to \( P(T > 200 - 150) = P(T > 50) \).

Now, we calculate \( P(T > 50) \) for each machine:

For the first machine with \( \lambda_1 = 0.01 \):

\[
P(T_1 > 50) = e^{-0.01 \times 50} = e^{-0.5} \approx 0.6065.
\]

For the second machine with \( \lambda_2 = 0.005 \):

\[
P(T_2 > 50) = e^{-0.005 \times 50} = e^{-0.25} \approx 0.7788.
\]

Since 1/3 of the transistors come from the first machine and 2/3 come from the second machine, the weighted probability is:

\[
P(T > 50) = \frac{1}{3} P(T_1 > 50) + \frac{2}{3} P(T_2 > 50)
= \frac{1}{3} \times 0.6065 + \frac{2}{3} \times 0.7788.
\]

Carrying out the calculation:

\[
P(T > 50) = 0.7214.
\]

Thus, the probability that the lifetime exceeds 200 hours, given that it has already exceeded 150 hours, is approximately:

\[
P(T > 200 \mid T > 150) \approx 0.7214.
\]

\section*{Problem 5}

Three students take equivalent standardized tests. 

\begin{itemize}
    \item On test 1, student 1 scores 144 with a mean of 128 and a standard deviation of 34.
    \item On test 2, student 2 scores 90 on a test with a mean of 86 and a standard deviation of 18.
    \item On test 3, student 3 scores 18 with a mean of 15 and a standard deviation of 5.
\end{itemize}

All the test scores are normally distributed. We are asked to determine which student's score is the most impressive. 

To compare the test scores, we can compute the z-scores for each student. The z-score is given by the formula:

\[
z = \frac{x - \mu}{\sigma},
\]

where \( x \) is the observed score, \( \mu \) is the mean of the test, and \( \sigma \) is the standard deviation.

\subsection*{Student 1}

For student 1, we have \( x = 144 \), \( \mu = 128 \), and \( \sigma = 34 \):

\[
z_1 = \frac{144 - 128}{34} = \frac{16}{34} \approx 0.47.
\]

\subsection*{Student 2}

For student 2, we have \( x = 90 \), \( \mu = 86 \), and \( \sigma = 18 \):

\[
z_2 = \frac{90 - 86}{18} = \frac{4}{18} \approx 0.22.
\]

\subsection*{Student 3}

For student 3, we have \( x = 18 \), \( \mu = 15 \), and \( \sigma = 5 \):

\[
z_3 = \frac{18 - 15}{5} = \frac{3}{5} = 0.6.
\]

\subsection*{Conclusion}

Comparing the z-scores of the three students:

\[
z_1 \approx 0.47, \quad z_2 \approx 0.22, \quad z_3 = 0.6.
\]

Since student 3 has the highest z-score (\( z_3 = 0.6 \)), student 3's score is the most impressive relative to the other students' scores.

\section*{Problem 6}

The time taken by a computer technician to fix a laptop is uniformly distributed between 15 minutes and 1 hour 15 minutes. 

The range of the uniform distribution is \( a = 15 \, \text{minutes} \) and \( b = 75 \, \text{minutes} \) (since 1 hour 15 minutes is 75 minutes).

\subsection*{(a) Find the probability that it takes the technician less than 30 minutes to fix the laptop.}

For a uniform distribution, the probability that a random variable \( X \) falls between two values is given by:

\[
P(a \leq X \leq b) = \frac{X - a}{b - a}.
\]

We are asked to find \( P(X < 30) \). Applying the formula:

\[
P(X < 30) = \frac{30 - 15}{75 - 15} = \frac{15}{60} = 0.25.
\]

Thus, the probability that it takes the technician less than 30 minutes is:

\[
P(X < 30) = 0.25.
\]

The corresponding R code to compute this probability using the `punif` function is:

```R
Probability of fixing the laptop in less than 30 minutes
punif(30, min = 15, max = 75)\

\subsection*{(b) Find the probability that it takes the technician between 45 minutes and one hour to fix the laptop.}

We are asked to find \( P(45 \leq X \leq 60) \), where 60 minutes corresponds to 1 hour. For a uniform distribution, the probability that a random variable \( X \) falls between two values is given by:

\[
P(a \leq X \leq b) = \frac{X_2 - X_1}{b - a}.
\]

Here, \( X_1 = 45 \, \text{minutes} \), \( X_2 = 60 \, \text{minutes} \), \( a = 15 \, \text{minutes} \), and \( b = 75 \, \text{minutes} \). Using this, we get:

\[
P(45 \leq X \leq 60) = \frac{60 - 45}{75 - 15} = \frac{15}{60} = 0.25.
\]

Thus, the probability that it takes the technician between 45 minutes and 1 hour to fix the laptop is:

\[
P(45 \leq X \leq 60) = 0.25.
\]

The corresponding R code to compute this probability using the `punif` function is:

Probability of fixing the laptop between 45 minutes and 60 minutes
punif(60, min = 15, max = 75) - punif(45, min = 15, max = 75)

\end{document}