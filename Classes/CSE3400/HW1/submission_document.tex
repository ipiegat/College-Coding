\documentclass{article} % Defines the type of document (e.g., article, report, book)
\usepackage[utf8]{inputenc} % Handles input encoding

\title{CSE3400 HW1} % Title of your document
\author{Isaac Piegat} % Author's name
\date{\today} % Date (you can also set it manually, like \date{September 10, 2024})

\begin{document} % Start of the document
\maketitle % Generates the title section

\section{Problem 1} % A section heading
I used python on a Windows 11 octa-core AMD Ryzen 9 8945HS. 

\subsection{Part 1} % A subsection
\begin{itemize}
    \item Time for $2^{10}$: 0.0 seconds 
    \item Time for $2^{20}$: 0.07856202125549316 seconds
    \item Time for $2^{30}$: 82.6774411201477 seconds
\end{itemize}

Estimated time for Part 1 ($2^{480}$): 89208.15517592432 seconds

\subsection{Part 2}
\begin{itemize}
    \item Time for $2^{10}$: 0.0 seconds
    \item Time for $2^{20}$: 0.13591599464416504 seconds
    \item Time for $2^{30}$: 139.3803141117096 seconds
\end{itemize}

Estimated time for Part 2 ($2^{480}$): 150382.65722608566 seconds

\section{Problem 2:}
Let \( G_0: \{0,1\}^n \to \{0,1\}^{2n} \) and \( G_1: \{0,1\}^{n/2} \to \{0,1\}^n \) be PRGs.

\subsection{Part 1: \(G_2(s \parallel t) = G_0(s) \parallel (t \oplus 1^n)\)}
The construction is NOT a PRG because of the vulnerability of \(t \oplus 1^n\). The xor operation only flips the bits of t, so the attacker would only need to guess t. As t is a random string with length n, there would be \(2^n\) possible combinations, thus the attacker would have a \(1/2^n\) probability of success. 

\subsection{Part 2: \(G_3(s \parallel z) = G_0(s \oplus z)\)}
The construction IS a PRG because of its unpredictability. The attacker, even if they knew s and z, would have no way to find the string produced by the pseudorandom operation of \(G_0\). The probability of guessing the binary string of length \(2n\) from \(G_3\) would be \(1/2^{2n}\).

\subsection{Part 3: \(G_4(s \parallel z) = LH(G_0(s) \oplus G_1(RH(z))\)), where LH is the left half of the input string, and RH is the right half of the input string}
The construction IS a PRG because of its unpredictability. \(G_0\) is a random binary string of length \(2n\), and \(G_1\) is a random binary string of length \(n\). (\(G_0\ \oplus G_1)\) would simply result in another random string of length \(n\), thus guessing would have a probability of \(1/2^{2n}\).

\subsection{Part 4: \(G_5(s)= (G_0(s) mod 2) \parallel G_0(s)\), where mod is the modulus operation.}
The construction IS NOT a PRG because it is a recognizable pattern. The first and second half of the output string with length \(4n\) would mirror each other because the xor operation does nothing to \(G_0\) since neither 0 or 1 have a remainder when divided by two, effectively not changing the string. As the attacker would only need to guess the first half of the string, the probability would be \(1/2^{2n}\). 

\section{Problem 3:}

\subsection{Part 1:}
\begin{enumerate}
    \item none
    \item none
\end{enumerate}
\end{document} % End of the document
