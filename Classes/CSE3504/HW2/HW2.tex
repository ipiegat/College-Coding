\documentclass{article}
\usepackage{graphicx}  % For including images
\usepackage{listings}   % For code formatting
\usepackage{xcolor}     % For colored code
\usepackage{caption}    % For caption formatting
\usepackage{csvsimple} % Package to import CSV files
\usepackage{booktabs}
\usepackage{float}
\usepackage{amsmath}


% Define R code settings for listings
\lstset{
    language=R,
    basicstyle=\ttfamily\footnotesize,
    keywordstyle=\color{blue},
    commentstyle=\color{green},
    stringstyle=\color{red},
    numbers=left,
    numberstyle=\tiny,
    stepnumber=1,
    breaklines=true,
    showstringspaces=false,
    frame=single,
    tabsize=2,
    captionpos=b
}

\title{CSE3504 Homework 2}
\author{Isaac Piegat}
\date{\today}

\begin{document}

\maketitle

\section{Problem 1:}
Serial numbers of 1000 laptops were examined and the first digit of each number was
noted. It was found that: (i) digit 1 occurs 305 times, (ii) digit 2 occurs 185 times, (iii) digit
3 occurs 124 times, (iv) digit 4 occurs 95 times, (v) digit 5 occurs 80 times, (vi) digit 6
occurs 64 times, (vii) digit 7 occurs 51 times, (viii) digit 8 occurs 49 times, and (ix) digit 9
occurs 47. Using R compute the probability mass function (PMF) and cumulative
distribution function (CDF). Display the PMF and CDF in R. Represent the PMF and CDF
in the form of a table. Plot the PMF and CDF. Please make sure you label the axes, and
the plots with appropriate titles. Include your R code, with answers with your submission. 

\subsection{R Code:}
\begin{lstlisting}[caption={R Code for PMF and CDF Calculation}]
digits <- 1:9
frequencies <- c(305, 185, 124, 95, 80, 64, 51, 49, 47)
total <- sum(frequencies)

pmf <- frequencies / total

cdf <- cumsum(pmf)

pmf_cdf_table <- data.frame(Digit = digits, Frequency = frequencies, PMF = pmf, CDF = cdf)

print(pmf_cdf_table)

png("pmf_plot.png")
plot(digits, pmf, type = "b", col = "blue", pch = 16, xlab = "Digits", ylab = "PMF", 
     main = "Probability Mass Function (PMF)", ylim = c(0, max(pmf)))
grid()
dev.off()

png("cdf_plot.png")
plot(digits, cdf, type = "b", col = "green", pch = 16, xlab = "Digits", ylab = "CDF", 
     main = "Cumulative Distribution Function (CDF)", ylim = c(0, 1))
grid()
dev.off()
\end{lstlisting}

\subsection{PMF and CDF Table}

\begin{table}[H]
\centering
\csvautotabular{"C:/Users/ipieg/OneDrive/Documents/pmf_cdf_table.csv"} % Reads and displays the CSV file
\caption{Probability Mass Function (PMF) and Cumulative Distribution Function (CDF) of Laptop Serial Numbers}
\end{table}

\subsection{PMF and CDF Plot}

\begin{figure}[H]
    \centering
    \includegraphics[width=0.6\textwidth]{"C:/Users/ipieg/OneDrive/Documents/pmf_plot.png"}
    \caption{Probability Mass Function (PMF) of the first digit of laptop serial numbers.}
    \label{fig:pmf}
\end{figure}

\begin{figure}[H]
    \centering
    \includegraphics[width=0.6\textwidth]{"C:/Users/ipieg/OneDrive/Documents/cdf_plot.png"}
    \caption{Cumulative Distribution Function (CDF) of the first digit of laptop serial numbers.}
    \label{fig:cdf}
\end{figure}

\section{Problem 2:}
Referring to Problem 1, we calculated the probabilities of different first digits of laptop
serial numbers. Suppose we are sampling laptop serial numbers at random, and when we
come across a serial number with a first digit as “9” we consider it a success. Write the
expressions for the following probabilities, and calculate them using the pbinom and
dbinom in R. Include your R code, with answers printed in the code.
a: If we sample 10 serial numbers, what is the probability that the number “9” will be the
first digit in at least 3 serial numbers? (9 points)
b: If we sample 10 serial numbers, what is the probability that the number “9” will not be
the first digit in any of the 10 numbers? (4 points)
c: What is the minimum number of serial numbers that need to be sampled if we want to
be at least .90 sure of finding at least one serial number with the first digit 9? In this
problem, we need to calculate the probability of at least one success for different sample
sizes. This problem involves experimentation with pbinom, and you need not write
mathematical expressions for this problem. Write a loop to calculate the probability of
at least one success for different sample sizes in R. Plot these probabilities as a function
of sample size. Using abline, place a horizontal line on the graph at 0.90 to graphically
identify the cutoff. Finally, find the minimum number of serial numbers. (22 points).

\subsection{Expressions:}
\textbf{a: Probability that the number “9” will be the first digit in at least 3 serial numbers}
\[
P(X \geq 3) = 1 - P(X \leq 2) = 1 - \sum_{k=0}^{2} \binom{n}{k} p^k (1 - p)^{n - k}
\]
where \( n = 10 \) and \( p = 0.047 \).
\newline
\newline
\textbf{b: Probability that the number “9” will not be the first digit in any of the 10 numbers}
\[
P(X = 0) = \binom{n}{0} p^0 (1 - p)^{n} = (1 - p)^{n}
\]
where \( n = 10 \) and \( p = 0.047 \).
\newline
\newline
\textbf{c: Minimum number of serial numbers to be at least 90\% sure of finding at least one serial number with the first digit 9}
\[
P(X \geq 1) = 1 - P(X = 0) = 1 - (1 - p)^n
\]
We solve for \( n \) such that \( 1 - (1 - p)^n \geq 0.90 \), where \( p = 0.047 \).

\subsection{R code:}
\begin{lstlisting}
    # Part (a)
    n <- 10
    p <- 0.047
    prob_at_least_3 <- 1 - pbinom(2, size = n, prob = p)
    print(prob_at_least_3) # answer is 0.009709584
    
    # Part (b)
    prob_zero_success <- dbinom(0, size = n, prob = p)
    print(prob_zero_success) # answer is 0.6179154
    
    # Part (c)
    p <- 0.047
    
    sample_sizes <- 1:100
    
    # Vector to store probabilities of at least one success for each sample size
    probs_at_least_one <- numeric(length(sample_sizes))
    
    # Loop to calculate the probability of at least one success for each sample size
    for (n in sample_sizes) {
      probs_at_least_one[n] <- 1 - pbinom(0, size = n, prob = p)
    }
    
    # Plot the probabilities as a function of sample size
    plot(sample_sizes, probs_at_least_one, type = "l", col = "blue", 
         xlab = "Sample Size", ylab = "Probability of At Least One Success", 
         main = "Probability of Finding at Least One Success")
    abline(h = 0.90, col = "red", lty = 2)  # Add a horizontal line at 0.90
    
    # Initialize variable for minimum sample size
    min_sample_size <- NULL
    prob_threshold <- 0.90  # 90% threshold
    
    # Find the first sample size where the probability exceeds 90%
    for (n in sample_sizes) {
      if (probs_at_least_one[n] >= prob_threshold) {
        min_sample_size <- n
        break  # Stop once we find the first sample size where the condition is met
      }
    }
    
    # Print the result
    print(min_sample_size) # answer is 48    
    \end{lstlisting}

\subsection{Part c plot:}
\begin{figure}[H]
    \centering
    \includegraphics[width=0.6\textwidth]{"C:/Users/ipieg/OneDrive/Documents/probability_plot.png"}
    \label{fig:prob_plot}
\end{figure}

\section{Problem 3:}
When Ava travels for professional commitments, the probability that she gains access to
the Internet is 0.6 at each attempt. She makes each attempt sequentially. Write the
expressions for the following probabilities, and compute them using the pgeom and
qgeom functions in R. Include the R code/commands, and the answer in the R code.
a: What is the probability that it takes no more than 4 attempts to get in? (6 points)
b: What is the probability that more than 4 attempts will be required to gain access? (5
points)
c: If Ava has already made two attempts without gaining access, what is the probability
that she will gain access in the next four attempts? (7 points)
d: What is the minimum number of attempts necessary to be at least .95 certain that Ava
will gain access? (7 points)
Write the expressions for the following probabilities, and compute them using the pgeom
and qgeom functions in R. Include the R code/commands, and the answer in the R code. 

\subsection{Expressions:}
textbf{a: Probability that it takes no more than 4 attempts to get in}

\[
P(X \leq 4) = \sum_{k=1}^{4} (1 - p)^{k-1} p = 1 - (1 - p)^4
\]
where \( p = 0.6 \).
\newline
\newline
\textbf{b: Probability that more than 4 attempts will be required to gain access}

\[
P(X > 4) = 1 - P(X \leq 4) = (1 - p)^4
\]
where \( p = 0.6 \).
\newline
\newline
\textbf{c: Probability that Ava will gain access in the next four attempts, given that she has already made two unsuccessful attempts}

\[
P(X \leq 4) = 1 - (1 - p)^4
\]
where \( p = 0.6 \). The previous two failed attempts do not affect this probability.
\newline
\newline
\textbf{d: Minimum number of attempts necessary to be at least 95\% certain that Ava will gain access}

\[
P(X \leq k) \geq 0.95
\]
This can be written as:
\[
k = \min \{ k: 1 - (1 - p)^k \geq 0.95 \}
\]
where \( p = 0.6 \).
\subsection{R code:}
\begin{lstlisting}
    # Part (a)
    p_success <- 0.6

    prob_no_more_than_4_attempts <- pgeom(3, prob = p_success)

    print(prob_no_more_than_4_attempts) # answer is 0.9744

    # Part (b)
    p_success <- 0.6

    # P(more than 4 attempts)= 1 - P(no more than 4 attempts)
    prob_more_than_4_attempts <- 1 - pgeom(3, prob = p_success)

    print(prob_more_than_4_attempts) # answer is 0.0256

    # Part (c)
    p_success <- 0.6

    prob_more_than_4_attempts <- 1 - pgeom(3, prob = p_success)

    print(prob_more_than_4_attempts) # answer is .9744 (same as part a)

    # Part (d)
    p_success <- 0.6

    target_prob <- 0.95
    min_attempts_for_95_percent <- qgeom(target_prob, prob = p_success)

    min_attempts_for_95_percent <- min_attempts_for_95_percent + 1

    print(min_attempts_for_95_percent) # answer is 4

\end{lstlisting}

\section{Problem 4:}
A new user of a smartphone receives an average of 3 messages per day. The arrival
pattern of these messages is Poisson. Calculate the probabilities that:
a: Exactly three messages will be received in any day? (6 points)
b: A day will pass without receiving any message? (5 points)
c: More than 3 messages will be received in any day? (9 points)
Write expressions for the following probabilities, and compute them using R using dpois
and ppois. Include your R code, along with your answers in the submission.

\subsection{Expressions:}
\textbf{a: Exactly three messages will be received in any day}

\[
P(X = 3) = \frac{\lambda^3 e^{-\lambda}}{3!}
\]
where \( \lambda = 3 \).
\newline
\newline
\textbf{b: A day will pass without receiving any message}

\[
P(X = 0) = \frac{\lambda^0 e^{-\lambda}}{0!} = e^{-\lambda}
\]
where \( \lambda = 3 \).
\newline
\newline
\textbf{c: More than 3 messages will be received in any day}

\[
P(X > 3) = 1 - P(X \leq 3) = 1 - \sum_{k=0}^{3} \frac{\lambda^k e^{-\lambda}}{k!}
\]
where \( \lambda = 3 \).

\subsection{R code:}
\begin{lstlisting}
    # Part (a)
    lambda <- 3  # average messages per day

    prob_exactly_3_messages <- dpois(3, lambda = lambda)

    print(prob_exactly_3_messages) # answer is 0.2240418

    # Part (b)
    lambda <- 3  # average messages per day

    prob_zero_messages <- dpois(0, lambda = lambda)

    print(prob_zero_messages) # answer is 0.04978707

    # Part (c)
    lambda <- 3  # average messages per day

    prob_more_than_3_messages <- 1 - ppois(3, lambda = lambda)

    print(prob_more_than_3_messages) # answer is 0.3527681

\end{lstlisting}

\section{Problem 5:}
Suppose .01 of chips in a large batch on an assembly line are defective. What is the
chance that from a sample 100 chips picked at random from this population, at least four
will be defective? Write the expressions to compute this probability using a: Binomial
distribution and b: Poisson approximation. Compute the final answers using pbinom, and
ppois functions in R. Include the R commands and the answers in your submission.
\subsection{Expressions:}
\textbf{Binomial distribution:}
\[
P(X \geq 4) = 1 - P(X \leq 3) = 1 - \sum_{k=0}^{3} \binom{n}{k} p^k (1-p)^{n-k}
\]
where \( n = 100 \) and \( p = 0.01 \).
\newline
\newline
\textbf{Poisson approximation:}
\[
P(X \geq 4) = 1 - P(X \leq 3) = 1 - \sum_{k=0}^{3} \frac{\lambda^k e^{-\lambda}}{k!}
\]
where \( \lambda = n \times p = 1 \).
\subsection{R code:}
\begin{lstlisting}
    # Given data
    n <- 100  # Sample size
    p <- 0.01  # Probability of a defective chip
    lambda <- n * p  # Poisson parameter (lambda)

    # Binomial distribution
    prob_binom <- 1 - pbinom(3, size = n, prob = p)
    print(prob_binom)  # answer is 0.01837404

    #  Poisson approximation
    prob_poisson <- 1 - ppois(3, lambda = lambda)
    print(prob_poisson)  # answer is 0.01837404
\end{lstlisting}

\section{Problem 6:}
Let X and Y be two random draws from a box containing three numbers {1, 2, 3}.
Calculate the distribution of S = X + Y, when X and Y are drawn a: without replacement,
and b: with replacement.
\subsection{Without replacement:}
Combinations: (1,2),(1,3),(2,1),(2,3),(3,1),(3,2)
\newline
Sums: (3, 4, 3, 5, 4, 5)
\newline
Probabilities: S = 3, S = 4, and S = 5 all occur $2/6$ times, thus each has a probability of $P = 1/3$.

\subsection{With replacement:}
Combinations: (1,1),(1,2),(1,3),(2,1),(2,2),(2,3),(3,1),(3,2),(3,3)
\newline
Sums: (2, 3, 4, 3, 4, 5, 4, 5, 6)
\newline
Probabilities: $P(S=2) = 1/9$, $P(S=3) = 2/9$, $P(S=4) = 3/9$, $P(S=5) = 2/9$, $P(S=6) = 1/9$.
\end{document}
